\subsection{External interfaces}
\subsubsection{User interfaces}
\subsubsection{Hardware interfaces}
\subsubsection{Software interfaces}

\subsection{Functional Requirements}
\subsubsection{Use cases diagrams}
\subsubsection{Sequence diagrams}
\subsubsection{Requirements mapping}
\comment{Da fare alla fine}

\subsection{Performance requirements}
\begin{enumerate}
  \item \textbf{Response time:} \\
  To ensure timely responsiveness for user interactions the platform should respond to user actions (e.g., loading pages, submitting forms) within a maximum of two seconds under normal load conditions.
  \item \textbf{Scalability:} \\
  To ensure the platform can handle an increasing number of users and data the platform should support concurrent access by at least 1000 users without significant degradation in performance, for up to 1000 battles simultaneously.
  \item \textbf{GitHub integration:} \\
  GitHub repository creation, including code kata and automation setup, should take no longer than two minutes, also automated workflow triggers from GHA should result in platform updates within 1 minute of code submission, otherwise some students’ submission may be lost.
  \item \textbf{Automated assessment and consolidation stage:} \\
  To ensure fast evaluation of code submissions, automated assessments (including code analysis and test execution) should complete within two minutes for a battle. Also, the platform should handle simultaneous automated assessments from multiple battles without queuing delays.
  \item \textbf{Notification system:} \\
  Notifications about battle and tournament updates need to be sent to every interested user in at most one minute.
  \item \textbf{Badges and gamification:} \\
  To provide a continuous experience for badge assignment and visualization, they need to be assigned right after the end of each tournament and to the correct students. Also, their visualization on users’ profile should load instantaneously.
  \item \textbf{Security:} \\
  To ensure secure data handling and prevent unauthorized access user authentication and authorization processes should happen rapidly and the platform should do regular security inspections to identify and address potential vulnerabilities.
  \item \textbf{Data storage and retrieval:} \\
  To optimize data storage and retrieval processes, database queries for common operations should be fully optimized. Frequent backups are done to prevent data losses.
  \item \textbf{Reliability:} \\
  To ensure consistent and reliable performance, the platform should achieve 99.9\% uptime.
\end{enumerate}
These are criteria that CKB aims to achieve. Rigorous testing and continuous monitoring will be essential to meet and maintain these standards throughout the system’s lifecycle.

\subsection{Design constraints}
These are the constraints related to the design of the system. They are divided in three
categories: standards compliance, hardware limitations and other constraints.

\subsubsection{Standard compliance}
\begin{enumerate}
  \item \textbf{Web standards:} \\ 
  User interface and interactions must adhere to modern web standards, ensuring compatibility with major web browsers such as Chrome, Firefox, Safari, and Edge.
  \item \textbf{Coding standards:} \\
  Code written for the platform must stick to standard coding conventions, such as proper documentation and modular design.
  \item \textbf{Security standards:} \\
  The platform must comply with standard security protocols to safeguard user data, prevent unauthorized access and protect against common web vulnerabilities, e.g. Cross-Site Scripting, File Inclusion Vulnerabilities or injection attacks.
  \item \textbf{Data privacy regulations:} \\
  The platform must comply with data privacy regulations  such as General Data Protection Regulation (GDPR) and California Consumer Privacy Act (CCPA), and ensure the secure handling of user data, in addition, explicit consent must be obtained from users regarding data collection and usage.
  \item \textbf{Environmental sustainability for servers:} \\
  Servers hosting the platform need to adhere to the most recent environmental sustainability standards. This includes considerations for energy efficiency, eco-friendly practices, and responsible disposal of electronic waste.
\end{enumerate}

\subsubsection{Hardware limitations}
\begin{enumerate}
  \item \textbf{Server requirements:} \\
  The platform's server infrastructure must meet the specified minimum requirements for processing power, memory and storage to ensure optimal performance and responsiveness.
  \item \textbf{Internet connection:} \\
  Users accessing the platform must have a reliable and high-speed internet connection to ensure efficient participation in battles, timely submission of solutions and access to real-time updates.
\end{enumerate}

\subsubsection{Any other constraints}
\begin{enumerate}
  \item \textbf{Educator access:} \\
  Educators must have appropriate permissions to create and manage tournaments, battles and associated configurations. Access control mechanisms must be in place via discretionary access control to prevent unauthorized access.
  \item \textbf{GitHub integration:} \\
  Users must have a GitHub account to fully utilize the platform's features, especially the integration with GitHub repositories and automated workflows.
  \item \textbf{GitHub API rate limits:} \\
  The platform must adhere to GitHub's API rate limits to avoid disruptions in GitHub integration. Excessive API requests may result in rate-limiting, affecting the automated workflow triggered by code submissions.
  \item \textbf{Scalability considerations:} \\
  Scaling beyond a certain threshold may lead to additional costs or require adjustments to the hosting environment.
\end{enumerate}

\subsection{Software System Attributes}
\subsubsection{Reliability}
The platform must consistently and accurately perform its functions without failures or errors. To do so is necessary to implement error handling mechanisms and use a redundant system to ensure continuous operation. \\
Also to not lose data in case of a failure it’s necessary to frequently back up all the data in the database, keeping them in a different building and offline to their security. \\
It’s also important to test the whole system, including unit testing and integration testing, to identify and correct bugs and malfunctioning.

\subsubsection{Availability}
The platform must be accessible to users when needed. A way to assure that is to implement a physical load balancing system that distributes incoming traffic through different server’s replicas to prevent overload. Server’s and data center’s replicas also permit to minimize downtime in case of hardware failures. The usage of monitoring tools can also be very helpful to detect and respond to issues, ensuring high system uptime. \\
Since this system does not provide emergency services or services related to critical situations, it must provide availability of 99.9\%. To have the best results, it is essential that the Mean Time To Failures (MTTF) is as long as possible and that the Mean Time to Repair (MTTR) is as short as possible.

\subsubsection{Security}
To safeguard the platform against unauthorized access, data breaches and malicious activities, it’s necessary to employ an encryption mechanism to protect data during transmission (avoiding attacks like sniffing and spoofing) and storage in the database, also to guarantee users’ privacy. Developers need to implement secure coding practices to mitigate vulnerabilities such as injection attacks (e.g. SQL injection), cross-site scripting (XSS) and cross-site request forgery (CSRF), at physical level a Web Application Firewall (WAF) is employed to protect against this common attacks and also to log and alert on suspicious activities, providing insights for threat response. \\
Also, since the script uploaded can be potentially dangerous to execute on servers, it’s necessary to implement a robust sandboxing mechanism to isolate and execute user-uploaded code securely, can also be very helpful the usage of static code analysis tools and threat intelligence feeds and pattern matching to check user-submitted code for potential security vulnerabilities or malicious code patterns before execution. Another important thing is applying restrictions on resource usage (CPU, memory) to prevent malicious activities from the users. \\
At physical level are necessary infrastructures such as: firewalls that filter and monitor incoming network traffic, Intrusion Detection System (IDS) to detect and respond to security threats or unusual activities. \\
Finally, frequently conduct regular checks and reviews for security and vulnerability and penetration testing to identify and address potential weaknesses is fundamental.

\subsubsection{Maintainability}
The platform should be easily modified, updated and widened over its lifecycle. Some approaches can be adopting modular code structures to facilitate code maintenance and utilizing version control systems for tracking changes and managing collaborative development, also it’s important the incorporation of an automated testing routine that covers at least the 75\% of the code and the continuous update of the software.

\subsubsection{Portability}
The platform needs to run on different environments and platforms. The principal solution is to implement platform-independent code by adhering to standard programming languages and frameworks, also it’s necessary to address dependencies and ensure compatibility with different operating systems and web browsers, this can be done using containerization technologies (e.g. Docker) for packaging and deploying applications consistently across various environments.